% This is LLNCS.DEM the demonstration file of
% the LaTeX macro package from Springer-Verlag
% for Lecture Notes in Computer Science,
% version 2.4 for LaTeX2e as of 16. April 2010
%
\documentclass{llncs}
\begin{document}

\title{ 15h International Conference on Computer Analysis of Images and Patterns\\
York, 27-29 August 2013}

%\titlerunning{article number 0}  % abbreviated title (for running head)
%                                     also used for the TOC unless
%                                     \toctitle is used
%
\author{author1\inst{1} \and author 2\inst{2}}
%
%\authorrunning{Ivar Ekeland et al.} % abbreviated author list (for running head)
%
%%%% list of authors for the TOC (use if author list has to be modified)
%\tocauthor{author1, author2}
%
\institute{Institute1,\\
\email{email1@email.com},
\and
Insitute2,\\
\email{email2@email.com}}

\maketitle              % typeset the title of the contribution

\begin{abstract}
This is a short example to show how to use llncs.cls to submit a paper to CAIP 2013.
\keywords{CAIP 2013; example}
\end{abstract}
%
\section{Introduction}
%
CAIP 2013 is the fiftheenth in the CAIP series of biennial international conferences devoted to all aspects of Computer Vision, Image Analysis and Processing, Pattern recognition and related fields. CAIP2013 will be hosted by York University and held in August 27-29,  in York, UK. Previous CAIP conferences were held in Seville (2011), Munster (2009), Vienna (2007), Paris (2005) and Groningen (2003). The scientific program of the conference will consist of several keynote addresses, high quality papers selected by the international program committee and presented in a single track. Poster presentations will allow expert discussions of specialized research topics. 
%\subsection{Autonomous Systems}
%
\section{Scope}

The scope of CAIP'13 includes, but not limited to, the following areas:
\begin{itemize}
%%%%%%%%%%%%%%%%%%%%%%%%%%%%%%%
\item 3D TV

\item 3D Vision

\item Applications

\item Biometrics

\item Calibration

\item Color and texture

\item Curves, surfaces and objects beyond 2 dimensions

\item Document Analysis

\item Geo-topological analysis of images

\item Graph-based Methods

\item Image and video indexing and database retrieval

\item Image and video processing

\item Image Registration

\item Image-based modeling

\item Kernel methods

\item Medical imaging

\item Mobile multimedia

\item Model-based vision approaches

\item Motion Analysis

\item Natural computation for digital imagery

\item Non-photorealistic animation and modeling

\item Object recognition

\item Performance evaluation

\item Segmentation and grouping

\item Shape Recovery

\item Shape representation and analysis

\item Stereo and video analysis

\item Structural pattern recognition

\item Texture analysis

\item Tracking

\item Others


%%%%%%%%%%%%%%%%%%%%%%%%%%%%%%%
%%\item 3D Vision
%%\item 3D TV
%%\item Biometrics
%%\item Color and texture
%%\item Document analysis
%%\item Graph-based Methods
%%\item Image and video indexing and database retrieval
%%\item Image and video processing
%%\item Image-based modeling
%%\item Kernel methods
%%\item Medical imaging
%%\item Mobile multimedia
%%\item Model-based vision approaches
%%\item Motion Analysis
%%\item Non-photorealistic animation and modeling
%%\item Object recognition
%%\item Performance evaluation
%%\item Segmentation and grouping
%%\item Shape representation and analysis
%%\item Structural pattern recognition
%%\item Tracking
%%\item Applications
\end{itemize}

\section{Publication}
The conference proceedings will be published in the  Springer LNCS  series.

\section{Third section}
Here the sections required.


\section{Fourth section}
Here the sections required.

...............
%
\begin{thebibliography}{}
%
\bibitem{firstbook}
Author1, A., Auhtor2, A.:
the title of the article.
Journal 78, 315--333 (1982)

\end{thebibliography}
\end{document}

